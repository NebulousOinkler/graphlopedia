%%%%% Graphlopedia Source File 
 %%% Do not edit here!  
 %%% Edit the master json file directly and rerun WriteAllGraphsToSource.

\graphname{G000001}
\graphtitle{Triangle}
\degrees{[2, 2, 2]}
\vertices{3}
\edges{[[1, 2], [1, 3], [2, 3]]}
\pictures{G000001.png}
\comments 
  \begin{enumerate} 
 
    \item Complete graph on 3 vertices, $K_3$, 
    \item Coxeter graph of type affine $A_2$, 
    \item Cycle on 3 vertices, $C_3$, 
  \end{enumerate} 
 
\links 
  \begin{enumerate} 
    \item \url{http://mathworld.wolfram.com/TriangleGraph.html}, 
    \item \url{https://en.wikipedia.org/wiki/Triangle_graph}, 
  \end{enumerate} 
 
\refs 
  \begin{enumerate} 
    \item J. E. Humphreys, Reflection Groups and Coxeter Groups, Cambridge Studies in Advanced Math, Volume 29, 1990. Page 34., 
  \end{enumerate} 
 
\owner{Sara Billey.}

\graphname{G000002}
\graphtitle{2-Path, $P_2$}
\degrees{[1, 1]}
\vertices{2}
\edges{[[1, 2]]}
\pictures{G000002.png}
\comments 
  \begin{enumerate} 
 
    \item Coxeter graph of type $A_2$, 
  \end{enumerate} 
 
\links 
  \begin{enumerate} 
    \item \url{https://en.wikipedia.org/wiki/Path_graph}, 
  \end{enumerate} 
 
\refs 
  \begin{enumerate} 
    \item J. E. Humphreys, Reflection Groups and Coxeter Groups, Cambridge Studies in Advanced Math, Volume 29, 1990. Page 32., 
  \end{enumerate} 
 
\owner{Sara Billey.}

\graphname{G000003}
\graphtitle{3-Path, $P_3$}
\degrees{[2, 1, 1]}
\vertices{3}
\edges{[[1, 2], [1, 3]]}
\pictures{G000003.png}
\comments 
  \begin{enumerate} 
 
    \item Coxeter graph of type $A_3$, 
  \end{enumerate} 
 
\links 
  \begin{enumerate} 
    \item \url{https://en.wikipedia.org/wiki/Path_graph}, 
  \end{enumerate} 
 
\refs 
  \begin{enumerate} 
    \item J. E. Humphreys, Reflection Groups and Coxeter Groups, Cambridge Studies in Advanced Math, Volume 29, 1990. Page 32., 
  \end{enumerate} 
 
\owner{Sara Billey.}

\graphname{G000004}
\graphtitle{4-Path, $P_4$}
\degrees{[2, 2, 1, 1]}
\vertices{4}
\edges{[[1, 2], [1, 4], [2, 3]]}
\pictures{G000004.png}
\comments 
  \begin{enumerate} 
 
    \item Coxeter graph of type $A_4$, 
  \end{enumerate} 
 
\links 
  \begin{enumerate} 
    \item \url{https://en.wikipedia.org/wiki/Path_graph}, 
  \end{enumerate} 
 
\refs 
  \begin{enumerate} 
    \item J. E. Humphreys, Reflection Groups and Coxeter Groups, Cambridge Studies in Advanced Math, Volume 29, 1990. Page 32., 
  \end{enumerate} 
 
\owner{Sara Billey.}

\graphname{G000005}
\graphtitle{5-Path, $P_5$}
\degrees{[2, 2, 2, 1, 1]}
\vertices{5}
\edges{[[1, 2], [1, 5], [2, 3], [3, 4]]}
\pictures{G000005.png}
\comments 
  \begin{enumerate} 
 
    \item Coxeter graph of type $A_5$, 
  \end{enumerate} 
 
\links 
  \begin{enumerate} 
    \item \url{https://en.wikipedia.org/wiki/Path_graph}, 
  \end{enumerate} 
 
\refs 
  \begin{enumerate} 
    \item J. E. Humphreys, Reflection Groups and Coxeter Groups, Cambridge Studies in Advanced Math, Volume 29, 1990. Page 32., 
  \end{enumerate} 
 
\owner{Sara Billey.}

\graphname{G000006}
\graphtitle{Claw}
\degrees{[3, 1, 1, 1]}
\vertices{4}
\edges{[[1, 2], [1, 3], [1, 4]]}
\pictures{G000006.png}
\comments 
  \begin{enumerate} 
 
    \item star graph of type $(1,3)$, 
    \item complete bipartite graph $K_{1,3}$, 
    \item Coxeter graph of type $D_4$, 
  \end{enumerate} 
 
\links 
  \begin{enumerate} 
    \item \url{http://mathworld.wolfram.com/ClawGraph.html}, 
  \end{enumerate} 
 
\refs 
  \begin{enumerate} 
    \item Horton, J. D. and Bouwer, I. Z. Symmetric Y-Graphs and H-Graphs. J. Combin. Th. Ser. B 53, (1991). Page 116., 
    \item Humphreys J., Reflection Groups and Coxeter Groups, Cambridge Studies in Advanced Math, Volume 29, 1990. Page 32., 
    \item Dahlberg, S., Foley, A., and van Willigenburg, S.  Resolving Stanley's e-positivity of claw contractible free graphs. Preprint arXiv:1703.05770, (2017), Page 5., 
    \item Gasharov V., On Stanley's chromatic symmetric function and clawfree graphs, Discrete Math. 205, 229-234 (1999)., 
  \end{enumerate} 
 
\owner{Aaron Bode.}

\graphname{G000007}
\graphtitle{3-Antiprism}
\degrees{[4, 4, 4, 4, 4, 4]}
\vertices{6}
\edges{[[1, 2], [1, 3], [1, 4], [1, 5], [2, 3], [2, 5], [2, 6], [3, 4], [3, 6], [4, 5], [4, 6], [5, 6]]}
\pictures{G000007.png}
\comments 
  \begin{enumerate} 
 
    \item planar, 
  \end{enumerate} 
 
\links 
  \begin{enumerate} 
    \item \url{http://mathworld.wolfram.com/AntiprismGraph.html}, 
    \item \url{https://en.wikipedia.org/wiki/Antiprism_graph}, 
  \end{enumerate} 
 
\refs 
  \begin{enumerate} 
    \item Alekseyev, M.; Michon, G. Making Walks Count: From Silent Circles to Hamiltonian Cycles. eprint arXiv:1602.01396. (2016), 
  \end{enumerate} 
 
\owner{Aaron Bode.}

\graphname{G000008}
\graphtitle{3-Barbell}
\degrees{[3, 3, 2, 2, 2, 2]}
\vertices{6}
\edges{[[1, 2], [1, 3], [1, 4], [2, 5], [2, 6], [3, 4], [5, 6]]}
\pictures{G000008.png}
\comments 
  \begin{enumerate} 
 
    \item planar, 
  \end{enumerate} 
 
\links 
  \begin{enumerate} 
    \item \url{http://mathworld.wolfram.com/BarbellGraph.html}, 
    \item \url{https://en.wikipedia.org/wiki/Barbell_graph}, 
  \end{enumerate} 
 
\refs 
  \begin{enumerate} 
    \item Wilf, H. The editor's corner: the white screen problem. Amer. Math. Monthly 96 (1989), no. 8, 704--707. , 
  \end{enumerate} 
 
\owner{Aaron Bode.}

\graphname{G000009}
\graphtitle{Pappus Graph}
\degrees{[3, 3, 3, 3, 3, 3, 3, 3, 3, 3, 3, 3, 3, 3, 3, 3, 3, 3]}
\vertices{18}
\edges{[[1, 2], [1, 6], [1, 7], [2, 3], [2, 8], [3, 4], [3, 9], [4, 5], [4, 10], [5, 6], [5, 11], [6, 12], [7, 14], [7, 18], [8, 13], [8, 15], [9, 14], [9, 16], [10, 15], [10, 17], [11, 16], [11, 18], [12, 13], [12, 17], [13, 16], [14, 17], [15, 18]]}
\pictures{G000009.png}
\pictures{extra/G000009.png}
\comments 
  \begin{enumerate} 
 
    \item The Pappus graph is formed as the Levi graph of the Pappus configuration., 
  \end{enumerate} 
 
\links 
  \begin{enumerate} 
    \item \url{https://en.wikipedia.org/wiki/Pappus_graph}, 
    \item \url{http://mathworld.wolfram.com/PappusGraph.html}, 
  \end{enumerate} 
 
\refs 
  \begin{enumerate} 
    \item Coxeter, H. S. M. Self-Dual Configurations and Regular Graphs. Bull. Amer. Math. Soc. 56, 413-455, 1950. Page 434., 
  \end{enumerate} 
 
\owner{Sara Billey.}

\graphname{G000010}
\graphtitle{8-Vertex Threshold Graph}
\degrees{[7, 5, 2, 2, 2, 2, 1, 1]}
\vertices{8}
\edges{[[1, 2], [1, 3], [1, 4], [1, 5], [1, 6], [1, 7], [1, 8], [2, 3], [2, 4], [2, 5], [2, 6]]}
\pictures{G000010.png}
\comments 
  \begin{enumerate} 
 
    \item threshold, planar, and trivially perfect., 
  \end{enumerate} 
 
\links 
  \begin{enumerate} 
    \item \url{https://en.wikipedia.org/wiki/Threshold_graph}, 
  \end{enumerate} 
 
\refs 
  \begin{enumerate} 
    \item Heggernes, P.; Kratsch, D. Linear-time certifying recognition algorithms and forbidden induced subgraphs, Nordic Journal of Computing, 14 (1-2): 87-108 (2008), 
  \end{enumerate} 
 
\owner{Katrina Warner.}

\graphname{G000011}
\graphtitle{6-Vertex Circular-Arc}
\degrees{[3, 3, 2, 2, 2, 2]}
\vertices{6}
\edges{[[1, 2], [1, 3], [1, 4], [2, 3], [2, 5], [4, 6], [5, 6]]}
\pictures{G000011.png}
\comments 
  \begin{enumerate} 
 
    \item circular-arc graph, 
    \item intersection graph, 
    \item arc, 
  \end{enumerate} 
 
\links 
  \begin{enumerate} 
    \item \url{https://en.wikipedia.org/wiki/Circular-arc_graph}, 
  \end{enumerate} 
 
\owner{Katrina Warner.}

\graphname{G000012}
\graphtitle{4-Cycle}
\degrees{[2, 2, 2, 2]}
\vertices{4}
\edges{[[1, 2], [1, 4], [2, 3], [3, 4]]}
\pictures{G000012.png}
\comments 
  \begin{enumerate} 
 
    \item minimal non-trivially-perfect graph, 
  \end{enumerate} 
 
\links 
  \begin{enumerate} 
    \item \url{https://en.wikipedia.org/wiki/Trivially_perfect_graph}, 
  \end{enumerate} 
 
\refs 
  \begin{enumerate} 
    \item Martin Charles Golumbic, Trivially perfect graphs, Discrete Mathematics, Volume 24, Issue 1, 1978, Pages 105-107., 
  \end{enumerate} 
 
\owner{Katrina Warner, Sara Billey.}

\graphname{G000013}
\graphtitle{ladder graph}
\degrees{[3, 3, 3, 3, 2, 2, 2, 2]}
\vertices{8}
\edges{[[1, 2], [1, 4], [1, 6], [2, 3], [2, 8], [3, 4], [3, 7], [4, 5], [5, 6], [7, 8]]}
\pictures{G000013.png}
\comments 
  \begin{enumerate} 
 
    \item ladder graph, 
  \end{enumerate} 
 
\links 
  \begin{enumerate} 
    \item \url{https://en.wikipedia.org/wiki/Ladder_graph}, 
  \end{enumerate} 
 
\owner{Katrina Warner.}

\graphname{G000014}
\graphtitle{Truncated Icosahedral Graph}
\degrees{[3, 3, 3, 3, 3, 3, 3, 3, 3, 3, 3, 3, 3, 3, 3, 3, 3, 3, 3, 3, 3, 3, 3, 3, 3, 3, 3, 3, 3, 3, 3, 3, 3, 3, 3, 3, 3, 3, 3, 3, 3, 3, 3, 3, 3, 3, 3, 3, 3, 3, 3, 3, 3, 3, 3, 3, 3, 3, 3, 3]}
\vertices{60}
\edges{[[1, 2], [1, 6], [2, 3], [2, 9], [3, 4], [3, 12], [4, 5], [4, 15], [5, 1], [5, 18], [6, 7], [6, 20], [7, 8], [7, 21], [8, 9], [8, 24], [9, 10], [10, 11], [10, 25], [11, 12], [11, 28], [12, 13], [13, 14], [13, 29], [14, 15], [14, 32], [15, 16], [16, 17], [16, 33], [17, 18], [17, 36], [18, 19], [19, 20], [19, 37], [20, 40], [21, 22], [21, 40], [22, 23], [22, 41], [23, 24], [23, 43], [24, 25], [25, 26], [26, 27], [26, 44], [27, 28], [27, 46], [28, 29], [29, 30], [30, 31], [30, 47], [31, 32], [31, 49], [32, 33], [33, 34], [34, 35], [34, 50], [35, 36], [35, 52], [36, 37], [37, 38], [38, 39], [38, 53], [39, 40], [39, 55], [41, 42], [41, 55], [42, 43], [42, 56], [43, 44], [44, 45], [45, 46], [45, 57], [46, 47], [47, 48], [48, 49], [48, 58], [49, 50], [50, 51], [51, 52], [51, 59], [52, 53], [53, 54], [54, 55], [54, 60], [56, 57], [56, 60], [57, 58], [58, 59], [59, 60]]}
\pictures{G000014.png}
\comments 
  \begin{enumerate} 
 
    \item buckyball graph, 
    \item cayley, 
    \item 60-fullerene, 
  \end{enumerate} 
 
\links 
  \begin{enumerate} 
    \item \url{http://m.wolframalpha.com/input/?i=buckyball+graph}, 
  \end{enumerate} 
 
\refs 
  \begin{enumerate} 
    \item "Truncated Icosahedral Graph." Buckyball Graph -- from Wolfram MathWorld, $m.wolframalpha.com/input/?i=buckyball+graph$., 
  \end{enumerate} 
 
\owner{Aaron Bode.}

\graphname{G000015}
\graphtitle{$F_2$}
\degrees{[4, 4, 4, 4, 2, 2]}
\vertices{6}
\edges{[[1, 2], [1, 3], [1, 4], [1, 5], [2, 3], [2, 4], [2, 6], [3, 4], [3, 6], [4, 5]]}
\pictures{G000015.png}
\pictures{extra/G000015.png}
\comments 
  \begin{enumerate} 
 
    \item canonical example of a graph with an Eulerian cycle, 
  \end{enumerate} 
 
\refs 
  \begin{enumerate} 
    \item N. Chiarelli, Martin Milanic, A threshold approach to connected domination, University of Primorska, 2016. Page 4., 
  \end{enumerate} 
 
\owner{Aaron Bode.}

\graphname{G000016}
\graphtitle{5-Wheel Graph}
\degrees{[4, 3, 3, 3, 3]}
\vertices{5}
\edges{[[1, 2], [1, 3], [1, 4], [1, 5], [2, 3], [2, 5], [3, 4], [4, 5]]}
\pictures{G000016.png}
\comments 
  \begin{enumerate} 
 
    \item wheel graph $W_5$, 
  \end{enumerate} 
 
\links 
  \begin{enumerate} 
    \item \url{http://mathworld.wolfram.com/WheelGraph.html}, 
    \item \url{https://en.wikipedia.org/wiki/Wheel_graph}, 
  \end{enumerate} 
 
\owner{Katrina Warner.}

\graphname{G000017}
\graphtitle{Total Domishold Raft}
\degrees{[3, 3, 3, 3, 2, 2, 2, 2, 1, 1, 1, 1]}
\vertices{12}
\edges{[[1, 4], [1, 5], [1, 9], [2, 3], [2, 6], [2, 10], [3, 7], [3, 11], [4, 8], [4, 12], [5, 6], [7, 8]]}
\pictures{G000017.png}
\pictures{extra/G000017.png}
\comments 
  \begin{enumerate} 
 
    \item A total domishold graph that is not connected-domishold, 
  \end{enumerate} 
 
\links 
  \begin{enumerate} 
    \item \url{arxiv.org/pdf/1610.06539v1.pdf}, 
  \end{enumerate} 
 
\refs 
  \begin{enumerate} 
    \item Chiarelli, Nina, and Martin Milanic. "A Threshold Approach to Connected Domination." 21 Oct. 2016, arxiv.org/pdf/1610.06539v1.pdf., 
  \end{enumerate} 
 
\owner{Aaron Bode.}

\graphname{G000018}
\graphtitle{$G_k$ $K_{2,6}$ free}
\degrees{[9, 4, 4, 4, 3, 3, 3, 3, 3, 3, 3, 3, 3, 3, 3, 3, 3]}
\vertices{17}
\edges{[[1, 5], [1, 6], [1, 7], [1, 8], [1, 9], [1, 10], [1, 11], [1, 12], [1, 13], [2, 12], [2, 13], [2, 14], [2, 15], [3, 11], [3, 12], [3, 14], [3, 17], [4, 5], [4, 13], [4, 15], [4, 17], [5, 6], [6, 7], [7, 8], [8, 9], [9, 10], [10, 11], [14, 16], [15, 16], [16, 17]]}
\pictures{G000018.png}
\comments 
  \begin{enumerate} 
 
    \item 3-connected, 
    \item $G_k$ where $k\geq 1$, 
  \end{enumerate} 
 
\refs 
  \begin{enumerate} 
    \item Ellingham, M. N. et al. "Hamiltonicity of Planar Graphs with a Forbidden Minor." ArXiv, ArXiv, 20 Oct. 2016, arxiv.org/pdf/1610.06558v1.pdf., 
  \end{enumerate} 
 
\owner{Aaron Bode.}

\graphname{G000019}
\graphtitle{hexahedral graph 3}
\degrees{[4, 4, 3, 3, 3, 3]}
\vertices{6}
\edges{[[1, 2], [1, 3], [1, 4], [1, 6], [2, 3], [2, 4], [2, 5], [3, 5], [4, 6], [5, 6]]}
\pictures{G000019.png}
\comments 
  \begin{enumerate} 
 
    \item polyhedral graph, 
  \end{enumerate} 
 
\links 
  \begin{enumerate} 
    \item \url{http://mathworld.wolfram.com/PolyhedralGraph.html}, 
  \end{enumerate} 
 
\refs 
  \begin{enumerate} 
    \item "Polyhedral Graph." Polyhedral Graph -- from Wolfram MathWorld, $mathworld.wolfram.com/PolyhedralGraph.html$., 
  \end{enumerate} 
 
\owner{Aaron Bode.}

\graphname{G000020}
\graphtitle{Herschel Graph}
\degrees{[4, 4, 4, 3, 3, 3, 3, 3, 3, 3, 3]}
\vertices{11}
\edges{[[1, 4], [1, 6], [1, 7], [1, 11], [2, 6], [2, 7], [2, 8], [2, 9], [3, 4], [3, 8], [3, 9], [3, 11], [4, 5], [5, 6], [5, 8], [7, 10], [9, 10], [10, 11]]}
\pictures{G000020.png}
\comments 
  \begin{enumerate} 
 
    \item smallest nonhamiltonion polyhedral graph, 
    \item planar, 
  \end{enumerate} 
 
\links 
  \begin{enumerate} 
    \item \url{http://mathworld.wolfram.com/HerschelGraph.html}, 
  \end{enumerate} 
 
\refs 
  \begin{enumerate} 
    \item "Herschel Graph." Herschel Graph -- from Wolfram MathWorld, $mathworld.wolfram.com/HerschelGraph.html$., 
  \end{enumerate} 
 
\owner{Aaron Bode.}

\graphname{G000021}
\graphtitle{3-regular graph 1}
\degrees{[3, 3, 3, 3, 3, 3]}
\vertices{6}
\edges{[[1, 2], [1, 3], [1, 5], [2, 4], [2, 6], [3, 4], [3, 5], [4, 6], [5, 6]]}
\pictures{G000021.png}
\refs 
  \begin{enumerate} 
    \item Kong, Qi, and Ligong Wang. "The Signless Laplacian Spectral Radius of Subgraphs of Regular Graphs." https://Arxiv.org/Pdf/1610.08855v1.Pdf, ArXiv, 28 Oct. 2016, arxiv.org/pdf/1610.08855v1.pdf., 
  \end{enumerate} 
 
\owner{Aaron Bode.}

\graphname{G000022}
\graphtitle{3-regular graph 2}
\degrees{[3, 3, 3, 3, 3, 3, 3, 3]}
\vertices{8}
\edges{[[1, 2], [1, 3], [1, 5], [2, 4], [2, 6], [3, 4], [3, 7], [4, 8], [5, 6], [5, 7], [6, 8], [7, 8]]}
\pictures{G000022.png}
\refs 
  \begin{enumerate} 
    \item Q. Kong, L. Wang,The signless Laplacian spectral radius of subgraphs of regular graphs, Department of Applied Mathematics, School of Science, Northwestern Polytechnical University, 2016., 
  \end{enumerate} 
 
\owner{Aaron Bode.}

\graphname{G000023}
\graphtitle{5-Cycle, $C_5$}
\degrees{[2, 2, 2, 2, 2]}
\vertices{5}
\edges{[[1, 2], [1, 5], [2, 3], [3, 4], [4, 5]]}
\pictures{G000023.png}
\pictures{extra/G000023.png}
\links 
  \begin{enumerate} 
    \item \url{https://en.wikipedia.org/wiki/Cycle_graph}, 
  \end{enumerate} 
 
\owner{Zachary Hamaker.}

\graphname{G000024}
\graphtitle{Isobutane Molecule}
\degrees{[4, 4, 4, 4, 1, 1, 1, 1, 1, 1, 1, 1, 1, 1]}
\vertices{14}
\edges{[[1, 2], [1, 5], [1, 6], [1, 7], [2, 3], [2, 4], [2, 14], [3, 8], [3, 9], [3, 10], [4, 11], [4, 12], [4, 13]]}
\pictures{G000024.png}
\comments 
  \begin{enumerate} 
 
    \item connected forest, 
    \item tree graph, 
    \item Error in pubchem link, I don't see the graph there, but it is about isobutane mol--SB, 
  \end{enumerate} 
 
\links 
  \begin{enumerate} 
    \item \url{http://mathworld.wolfram.com/Tree.html}, 
    \item \url{https://pubchem.ncbi.nlm.nih.gov/compound/isobutane}, 
  \end{enumerate} 
 
\refs 
  \begin{enumerate} 
    \item  Tree -- from Wolfram MathWorld, mathworld.wolfram.com/Tree.html., 
  \end{enumerate} 
 
\owner{Aaron Bode.}

\graphname{G000025}
\graphtitle{Projective embedding of the positive roots of type $A_3$}
\degrees{[4, 3, 3, 2, 2, 2]}
\vertices{6}
\edges{[[1, 2], [1, 3], [1, 4], [1, 5], [2, 4], [2, 6], [3, 5], [3, 6]]}
\pictures{G000025.png}
\comments 
  \begin{enumerate} 
 
    \item Vertices are the intersections of the lines generated by the positive roots with a certain affine hyperplane.   Lines represent two dimensional spans., 
  \end{enumerate} 
 
\refs 
  \begin{enumerate} 
    \item S. Billey and A. Postnikov. Smoothness of Schubert varieties via patterns in root subsystems.  Advances in Applied Mathematics, vol 34 (2005).  Page 453., 
  \end{enumerate} 
 
\owner{Aaron Bode.}

\graphname{G000026}
\graphtitle{Heawood Graph}
\degrees{[3, 3, 3, 3, 3, 3, 3, 3, 3, 3, 3, 3, 3, 3]}
\vertices{14}
\edges{[[1, 2], [1, 10], [1, 14], [2, 3], [2, 7], [3, 4], [3, 12], [4, 5], [4, 9], [5, 6], [5, 14], [6, 7], [6, 11], [7, 8], [8, 9], [8, 13], [9, 10], [10, 11], [11, 12], [12, 13], [13, 14]]}
\pictures{G000026.png}
\comments 
  \begin{enumerate} 
 
    \item cage graph, 
    \item non-planar, 
  \end{enumerate} 
 
\links 
  \begin{enumerate} 
    \item \url{http://mathworld.wolfram.com/HeawoodGraph.html}, 
  \end{enumerate} 
 
\refs 
  \begin{enumerate} 
    \item Y. Zhao, Extremal regular graphs: independent sets and graph homomorphisms, 2016., 
  \end{enumerate} 
 
\owner{Aaron Bode.}

\graphname{G000027}
\graphtitle{Desargues' Graph}
\degrees{[3, 3, 3, 3, 3, 3, 3, 3, 3, 3, 3, 3, 3, 3, 3, 3, 3, 3, 3, 3]}
\vertices{20}
\edges{[[1, 2], [1, 6], [1, 20], [2, 3], [2, 17], [3, 4], [3, 12], [4, 5], [4, 15], [5, 6], [5, 10], [6, 7], [7, 8], [7, 16], [8, 9], [8, 19], [9, 10], [9, 14], [10, 11], [11, 12], [11, 20], [12, 13], [13, 14], [13, 18], [14, 15], [15, 16], [16, 17], [17, 18], [18, 19], [19, 20]]}
\pictures{G000027.png}
\pictures{extra/G000027.png}
\comments 
  \begin{enumerate} 
 
    \item cubic-symmetric graph, 
    \item Desargues graph is the first of four graphs depicted on the cover of Harary (1994)., 
  \end{enumerate} 
 
\links 
  \begin{enumerate} 
    \item \url{http://mathworld.wolfram.com/DesarguesGraph.html}, 
    \item \url{https://en.wikipedia.org/wiki/Desargues_graph}, 
  \end{enumerate} 
 
\refs 
  \begin{enumerate} 
    \item Kagno, I. N. Desargues' and Pappus' Graphs and Their Groups. Amer. J. Math. 69, 859-863, 1947., 
  \end{enumerate} 
 
\owner{Aaron Bode.}

\graphname{G000028}
\graphtitle{Figure 2(a)}
\degrees{[6, 3, 3, 3, 3, 3, 3]}
\vertices{7}
\edges{[[1, 2], [1, 3], [1, 4], [1, 5], [1, 6], [1, 7], [2, 3], [2, 7], [3, 4], [4, 5], [5, 6], [6, 7]]}
\pictures{G000028.png}
\links 
  \begin{enumerate} 
    \item \url{https://arxiv.org/pdf/math/0608624.pdf}, 
  \end{enumerate} 
 
\refs 
  \begin{enumerate} 
    \item W. Wood. Combinatorial Modulus and Types of Graphs, 2006., 
  \end{enumerate} 
 
\owner{Katrina Warner.}

\graphname{G000029}
\graphtitle{Figure 2(b)}
\degrees{[4, 4, 4, 2, 2, 2]}
\vertices{6}
\edges{[[1, 2], [1, 3], [1, 5], [1, 6], [2, 3], [2, 4], [2, 5], [3, 4], [3, 6]]}
\pictures{G000029.png}
\pictures{extra/G000029.png}
\links 
  \begin{enumerate} 
    \item \url{https://arxiv.org/pdf/math/0608624.pdf],}, 
  \end{enumerate} 
 
\refs 
  \begin{enumerate} 
    \item W. Wood. Combinatorial Modulus and Types of Graphs, 2006., 
  \end{enumerate} 
 
\owner{Katrina Warner.}

\graphname{G000030}
\graphtitle{threshold graph 2}
\degrees{[9, 9, 6, 6, 6, 5, 5, 2, 2, 2]}
\vertices{10}
\edges{[[1, 2], [1, 3], [1, 4], [1, 5], [1, 6], [1, 7], [1, 8], [1, 9], [1, 10], [2, 3], [2, 4], [2, 5], [2, 6], [2, 7], [2, 8], [2, 9], [2, 10], [3, 4], [3, 5], [3, 6], [3, 7], [4, 5], [4, 6], [4, 7], [5, 6], [5, 7]]}
\pictures{G000030.png}
\comments 
  \begin{enumerate} 
 
    \item threshold graph with binary string 0011100011, 
  \end{enumerate} 
 
\refs 
  \begin{enumerate} 
    \item A. Banerjee1, R. Mehatari. On the normalized spectrum of threshold graphs, Indian Institute of Science Education and Research Kolkata, 2016., 
  \end{enumerate} 
 
\owner{Aaron Bode.}

\graphname{G000031}
\graphtitle{Plabic Graph}
\degrees{[3, 3, 3, 3, 3, 3, 3, 3, 3, 3, 3, 3]}
\vertices{12}
\edges{[[1, 2], [1, 5], [1, 7], [2, 3], [2, 8], [3, 4], [3, 10], [4, 5], [4, 11], [5, 6], [6, 7], [6, 12], [7, 8], [8, 9], [9, 10], [9, 12], [10, 11], [11, 12]]}
\pictures{G000031.png}
\comments 
  \begin{enumerate} 
 
    \item plabic, 
    \item nonplanar, 
    \item undirected, 
  \end{enumerate} 
 
\links 
  \begin{enumerate} 
    \item \url{https://arxiv.org/pdf/1106.0023.pdf}, 
  \end{enumerate} 
 
\refs 
  \begin{enumerate} 
    \item Y. Kodoma, L. Williams. KP Solutions and Total Positivity for the Grassmannian, 2014., 
  \end{enumerate} 
 
\owner{Katrina Warner.}

\graphname{G000032}
\graphtitle{Figure 8}
\degrees{[3, 3, 3, 3, 3, 3, 3, 3, 3, 3, 3, 3, 3, 3, 3, 3, 3, 3, 3, 3]}
\vertices{20}
\edges{[[1, 2], [1, 3], [1, 10], [2, 8], [2, 15], [3, 6], [3, 11], [4, 7], [4, 9], [4, 14], [5, 6], [5, 7], [5, 17], [6, 16], [7, 18], [8, 9], [8, 20], [9, 19], [10, 11], [10, 15], [11, 12], [12, 13], [12, 17], [13, 14], [13, 16], [14, 15], [16, 20], [17, 18], [18, 19], [19, 20]]}
\pictures{G000032.png}
\comments 
  \begin{enumerate} 
 
    \item soliton, 
    \item plabic, 
  \end{enumerate} 
 
\links 
  \begin{enumerate} 
    \item \url{https://arxiv.org/pdf/1106.0023.pdf}, 
  \end{enumerate} 
 
\refs 
  \begin{enumerate} 
    \item Y. Kodoma, L. Williams. KP Solutions and Total Positivity for the Grassmannian, 2014., 
  \end{enumerate} 
 
\owner{Katrina Warner.}

\graphname{G000033}
\graphtitle{Figure 5}
\degrees{[4, 4, 4, 4, 4]}
\vertices{5}
\edges{[[1, 2], [1, 3], [1, 4], [1, 5], [2, 3], [2, 4], [2, 5], [3, 4], [3, 5], [4, 5]]}
\pictures{G000033.png}
\links 
  \begin{enumerate} 
    \item \url{https://arxiv.org/pdf/1106.0023.pdf}, 
  \end{enumerate} 
 
\refs 
  \begin{enumerate} 
    \item M. Han. Cosmological Constant in LQG Vertex Amplitude, 2011., 
  \end{enumerate} 
 
\owner{Katrina Warner.}

\graphname{G000034}
\graphtitle{6-cycle}
\degrees{[2, 2, 2, 2, 2, 2]}
\vertices{6}
\edges{[[1, 2], [1, 6], [2, 3], [3, 4], [4, 5], [5, 6]]}
\links 
  \begin{enumerate} 
    \item \url{https://en.wikipedia.org/wiki/Cycle_graph}, 
  \end{enumerate} 
 
\owner{Sara Billey.}

\graphname{G000035}
\graphtitle{Complete Bipartite Graph $K_{2,3}$}
\degrees{[3, 3, 2, 2, 2]}
\vertices{5}
\edges{[[1, 3], [1, 4], [1, 5], [2, 3], [2, 4], [2, 5]]}
\pictures{G000035.png}
\comments 
  \begin{enumerate} 
 
    \item The class of outerplanar graphs is closed under minor taking: its obstruction set consists of the graphs $K_{2,3}$ and $K_4$.  , 
  \end{enumerate} 
 
\refs 
  \begin{enumerate} 
    \item H.L. Bodlaender. A partial k-arboretum of graphs with bounded treewidth, Theoretical Computer Science 209. (1998). Page. 34, 
    \item M.M. Syslo, Characterisations of outerplanar graphs, Discrete Math. 26 (1979) 47-53., 
  \end{enumerate} 
 
\owner{Katrina Warner.}

\graphname{G000036}
\graphtitle{Guanine Structure}
\degrees{[3, 3, 3, 3, 2, 2, 2, 2, 2, 1, 1]}
\vertices{11}
\edges{[[1, 2], [1, 5], [1, 9], [2, 3], [2, 7], [3, 6], [3, 10], [4, 5], [4, 6], [4, 11], [7, 8], [8, 9]]}
\pictures{G000036.png}
\links 
  \begin{enumerate} 
    \item \url{https://arxiv.org/pdf/cs/0703132.pdf}, 
  \end{enumerate} 
 
\refs 
  \begin{enumerate} 
    \item L. Peshkin. Center for Biomedical Informatics, Harvard Medical School."Structure Induction by Lossless Graph Compression" (2007)., 
  \end{enumerate} 
 
\owner{Katrina Warner.}

\graphname{G000037}
\graphtitle{Sugar}
\degrees{[4, 3, 3, 2, 2, 2, 2, 1, 1, 1, 1]}
\vertices{11}
\edges{[[1, 4], [1, 9], [1, 10], [1, 11], [2, 3], [2, 4], [2, 5], [3, 7], [3, 8], [5, 6], [6, 7]]}
\pictures{G000037.png}
\pictures{extra/G000037.png}
\comments 
  \begin{enumerate} 
 
    \item The compound object induced by the Graphitour algorithm, which corresponds to the backbone of the molecule: phosphate and sugar., 
    \item Error in GRAPH: G000037 degree seq [4, 3, 3, 2, 2, 2, 2, 1, 1, 1, 1] should be [4, 3, 3, 2, 1, 0, 1, 1, 1, 1, 1]  [1, 2, 3, 4, 5, 6, 7, 8, 9, 10, 11] [[1, 4], [1, 9], [1, 10], [1, 11], [2, 3], [2, 4], [2, 5], [3, 7], [3, 8]], 
  \end{enumerate} 
 
\links 
  \begin{enumerate} 
    \item \url{https://arxiv.org/pdf/cs/0703132.pdf}, 
  \end{enumerate} 
 
\refs 
  \begin{enumerate} 
    \item L. Peshkin. Structure Induction by Lossless Graph Compression, (2007)., 
  \end{enumerate} 
 
\owner{Katrina Warner.}

\graphname{G000038}
\graphtitle{Original Factor Graph}
\degrees{[3, 3, 3, 3, 3, 3, 2, 2, 2, 2, 2, 2, 1, 1]}
\vertices{14}
\edges{[[1, 2], [1, 7], [1, 14], [2, 4], [2, 12], [3, 4], [3, 8], [3, 10], [4, 9], [5, 6], [5, 9], [5, 11], [6, 12], [6, 13], [7, 8], [10, 11]]}
\pictures{G000038.png}
\links 
  \begin{enumerate} 
    \item \url{https://arxiv.org/pdf/cs/0612030.pdf}, 
  \end{enumerate} 
 
\refs 
  \begin{enumerate} 
    \item J. Mooji, B. Kappen. "Loop Corrections for Approximate Inference" (2006), 
  \end{enumerate} 
 
\owner{Katrina Warner.}

\graphname{G000039}
\graphtitle{Cavity Graph of i}
\degrees{[3, 2, 2, 2, 2, 1, 1, 1, 1, 1]}
\vertices{10}
\edges{[[1, 2], [1, 8], [1, 9], [2, 3], [3, 10], [4, 5], [4, 7], [5, 6]]}
\pictures{G000039.png}
\links 
  \begin{enumerate} 
    \item \url{https://arxiv.org/pdf/cs/0612030.pdf}, 
  \end{enumerate} 
 
\refs 
  \begin{enumerate} 
    \item J. Mooji, B. Kappen. "Loop Corrections for Approximate Inference" (2006), 
  \end{enumerate} 
 
\owner{Katrina Warner.}

\graphname{G000040}
\graphtitle{Fig.1}
\degrees{[5, 5, 5, 5, 5, 5]}
\vertices{6}
\edges{[[1, 2], [1, 3], [1, 4], [1, 5], [1, 6], [2, 3], [2, 4], [2, 5], [2, 6], [3, 4], [3, 5], [3, 6], [4, 5], [4, 6], [5, 6]]}
\pictures{G000040.png}
\links 
  \begin{enumerate} 
    \item \url{https://arxiv.org/pdf/hep-th/0611042.pdf}, 
  \end{enumerate} 
 
\refs 
  \begin{enumerate} 
    \item A. Baratin, L. Friedel. Perimeter Institute for Theoretical Physics. "Hidden Quantum Gravity in 4d Feynman Diagrams" (2007), 
  \end{enumerate} 
 
\owner{Katrina Warner.}

\graphname{G000041}
\graphtitle{K5 Graph}
\degrees{[4, 4, 4, 4, 4]}
\vertices{5}
\edges{[[1, 2], [1, 3], [1, 4], [1, 5], [2, 3], [2, 4], [2, 5], [3, 4], [3, 5], [4, 5]]}
\pictures{G000041.png}
\links 
  \begin{enumerate} 
    \item \url{https://arxiv.org/pdf/hep-th/0611042.pdf}, 
  \end{enumerate} 
 
\refs 
  \begin{enumerate} 
    \item A. Baratin, L. Friedel. Perimeter Institute for Theoretical Physics. "Hidden Quantum Gravity in 4d Feynman Diagrams" (2007), 
  \end{enumerate} 
 
\owner{Katrina Warner.}

\graphname{G000042}
\graphtitle{Figure 1}
\degrees{[3, 2, 2, 1]}
\vertices{4}
\edges{[[1, 2], [1, 3], [1, 4], [2, 3]]}
\pictures{G000042.png}
\links 
  \begin{enumerate} 
    \item \url{https://arxiv.org/pdf/1304.0478.pdf}, 
  \end{enumerate} 
 
\refs 
  \begin{enumerate} 
    \item Z. Cinkir. "Explicit Computation of Certain Arakelov-Green Functions" (2013)., 
  \end{enumerate} 
 
\owner{Katrina Warner.}

\graphname{G000043}
\graphtitle{Tetrahedral Graph}
\degrees{[3, 3, 3, 3]}
\vertices{4}
\edges{[[1, 2], [1, 3], [1, 4], [2, 3], [2, 4], [3, 4]]}
\pictures{G000043.png}
\links 
  \begin{enumerate} 
    \item \url{https://arxiv.org/pdf/1304.0478.pdf}, 
  \end{enumerate} 
 
\refs 
  \begin{enumerate} 
    \item Z. Cinkir. "Explicit Computation of Certain Arakelov-Green Functions" (2013)., 
  \end{enumerate} 
 
\owner{Katrina Warner.}

\graphname{G000044}
\graphtitle{The staircase of order eight, St8}
\degrees{[3, 3, 3, 3, 3, 3, 3, 3]}
\vertices{8}
\edges{[[1, 2], [1, 3], [1, 4], [2, 3], [2, 8], [3, 5], [4, 5], [4, 6], [5, 7], [6, 7], [6, 8], [7, 8]]}
\pictures{G000044.png}
\pictures{extra/G000044.png}
\links 
  \begin{enumerate} 
    \item \url{https://arxiv.org/pdf/1611.07899.pdf}, 
  \end{enumerate} 
 
\refs 
  \begin{enumerate} 
    \item N. Kothari. "Generating Near-Bipartite Bricks" arXiv (2016). Page 4., 
  \end{enumerate} 
 
\owner{Katrina Warner.}

\graphname{G000045}
\graphtitle{Fano Graph}
\degrees{[3, 3, 3, 3, 3, 3, 3, 3, 3, 3, 3, 3, 3, 3]}
\vertices{14}
\edges{[[1, 2], [1, 6], [1, 14], [2, 3], [2, 11], [3, 4], [3, 8], [4, 5], [4, 13], [5, 6], [5, 10], [6, 7], [7, 8], [7, 12], [8, 9], [9, 10], [9, 14], [10, 11], [11, 12], [12, 13], [13, 14]]}
\pictures{G000045.png}
\pictures{extra/G000045.png}
\comments 
  \begin{enumerate} 
 
    \item The Fano graph is formed as the Levi graph of the Fano plane., 
  \end{enumerate} 
 
\links 
  \begin{enumerate} 
    \item \url{https://commons.wikimedia.org/wiki/File:Fano_plane-Levi_graph.svg}, 
    \item \url{https://en.wikipedia.org/wiki/Levi_graph}, 
    \item \url{https://en.wikipedia.org/wiki/Fano_plane}, 
  \end{enumerate} 
 
\refs 
  \begin{enumerate} 
    \item Coxeter, H. S. M. Self-Dual Configurations and Regular Graphs. Bull. Amer. Math. Soc. 56, 413-455, 1950. Page 424., 
  \end{enumerate} 
 
\owner{Sara Billey.}

\graphname{G000046}
\graphtitle{Complete Bipartite $K_5$}
\degrees{[4, 4, 4, 4, 4]}
\vertices{5}
\edges{[[1, 2], [1, 3], [1, 4], [1, 5], [2, 3], [2, 4], [2, 5], [3, 4], [3, 5], [4, 5]]}
\pictures{G000046.png}
\comments 
  \begin{enumerate} 
 
    \item Nonplanar, 
  \end{enumerate} 
 
\links 
  \begin{enumerate} 
    \item \url{https://en.wikipedia.org/wiki/Planar_graph}, 
  \end{enumerate} 
 
\owner{Aaron Bode.}

\graphname{G000047}
\graphtitle{Complete Bipartite $K_{3,3}$}
\degrees{[3, 3, 3, 3, 3, 3]}
\vertices{6}
\edges{[[1, 4], [1, 5], [1, 6], [2, 4], [2, 5], [2, 6], [3, 4], [3, 5], [3, 6]]}
\pictures{G000047.png}
\comments 
  \begin{enumerate} 
 
    \item Nonplanar, 
  \end{enumerate} 
 
\links 
  \begin{enumerate} 
    \item \url{https://en.wikipedia.org/wiki/Planar_graph}, 
    \item \url{http://mathworld.wolfram.com/NonplanarGraph.html}, 
  \end{enumerate} 
 
\owner{Aaron Bode.}

\graphname{G000048}
\graphtitle{Petersen Graph 1}
\degrees{[5, 4, 4, 4, 4, 3, 3, 3]}
\vertices{8}
\edges{[[1, 2], [1, 3], [1, 4], [1, 5], [1, 8], [2, 3], [2, 5], [2, 7], [3, 4], [3, 6], [4, 5], [4, 7], [5, 6], [6, 8], [7, 8]]}
\pictures{G000048.png}
\comments 
  \begin{enumerate} 
 
    \item Petersen graph, 
  \end{enumerate} 
 
\links 
  \begin{enumerate} 
    \item \url{https://en.wikipedia.org/wiki/Petersen_family}, 
  \end{enumerate} 
 
\owner{Aaron Bode.}

\graphname{G000049}
\graphtitle{Petersen Graph 2}
\degrees{[5, 5, 5, 4, 4, 4, 3]}
\vertices{7}
\edges{[[1, 2], [1, 3], [1, 4], [1, 5], [1, 6], [2, 3], [2, 4], [2, 5], [2, 6], [3, 4], [3, 5], [3, 6], [4, 7], [5, 7], [6, 7]]}
\pictures{G000049.png}
\comments 
  \begin{enumerate} 
 
    \item Petersen graph, 
  \end{enumerate} 
 
\links 
  \begin{enumerate} 
    \item \url{https://en.wikipedia.org/wiki/Petersen_family}, 
  \end{enumerate} 
 
\owner{Aaron Bode.}

\graphname{G000050}
\graphtitle{Petersen Graph 3}
\degrees{[4, 4, 4, 3, 3, 3, 3, 3, 3]}
\vertices{9}
\edges{[[1, 2], [1, 3], [1, 4], [1, 7], [2, 3], [2, 5], [2, 8], [3, 6], [3, 9], [4, 5], [4, 9], [5, 6], [6, 7], [7, 8], [8, 9]]}
\pictures{G000050.png}
\comments 
  \begin{enumerate} 
 
    \item Petersen graph, 
  \end{enumerate} 
 
\links 
  \begin{enumerate} 
    \item \url{https://en.wikipedia.org/wiki/Petersen_family}, 
  \end{enumerate} 
 
\owner{Aaron Bode.}

\graphname{G000051}
\graphtitle{Petersen Graph 4}
\degrees{[3, 3, 3, 3, 3, 3, 3, 3, 3, 3]}
\vertices{10}
\edges{[[1, 2], [1, 5], [1, 7], [2, 3], [2, 8], [3, 4], [3, 9], [4, 5], [4, 10], [5, 6], [6, 8], [6, 9], [7, 9], [7, 10], [8, 10]]}
\pictures{G000051.png}
\comments 
  \begin{enumerate} 
 
    \item Petersen graph, 
  \end{enumerate} 
 
\links 
  \begin{enumerate} 
    \item \url{https://en.wikipedia.org/wiki/Petersen_family}, 
  \end{enumerate} 
 
\refs 
  \begin{enumerate} 
    \item Y. Zhao, Extremal regular graphs: independent sets and graph homomorphisms, 2016., 
  \end{enumerate} 
 
\owner{Aaron Bode.}

\graphname{G000052}
\graphtitle{Petersen Graph 5}
\degrees{[6, 4, 4, 4, 4, 4, 4]}
\vertices{7}
\edges{[[1, 2], [1, 3], [1, 4], [1, 5], [1, 6], [1, 7], [2, 3], [2, 5], [2, 7], [3, 4], [3, 6], [4, 5], [4, 7], [5, 6], [6, 7]]}
\pictures{G000052.png}
\comments 
  \begin{enumerate} 
 
    \item Petersen graph, 
  \end{enumerate} 
 
\links 
  \begin{enumerate} 
    \item \url{https://en.wikipedia.org/wiki/Petersen_family}, 
  \end{enumerate} 
 
\owner{Aaron Bode.}

\graphname{G000053}
\graphtitle{Petersen Graph 6}
\degrees{[5, 5, 5, 5, 5, 5]}
\vertices{6}
\edges{[[1, 2], [1, 3], [1, 4], [1, 5], [1, 6], [2, 3], [2, 4], [2, 5], [2, 6], [3, 4], [3, 5], [3, 6], [4, 5], [4, 6], [5, 6]]}
\pictures{G000053.png}
\comments 
  \begin{enumerate} 
 
    \item Petersen graph, 
  \end{enumerate} 
 
\links 
  \begin{enumerate} 
    \item \url{https://en.wikipedia.org/wiki/Petersen_family}, 
  \end{enumerate} 
 
\owner{Aaron Bode.}

\graphname{G000054}
\graphtitle{Petersen Graph 7}
\degrees{[4, 4, 4, 4, 4, 4, 3, 3]}
\vertices{8}
\edges{[[1, 2], [1, 4], [1, 6], [1, 7], [2, 3], [2, 5], [2, 8], [3, 4], [3, 6], [3, 7], [4, 5], [4, 8], [5, 6], [5, 7], [6, 8]]}
\pictures{G000054.png}
\comments 
  \begin{enumerate} 
 
    \item Petersen graph, 
  \end{enumerate} 
 
\links 
  \begin{enumerate} 
    \item \url{https://en.wikipedia.org/wiki/Petersen_family}, 
  \end{enumerate} 
 
\owner{Aaron Bode.}

\graphname{G000055}
\graphtitle{Starfish}
\degrees{[3, 3, 3, 3, 3, 3, 3, 3, 3, 3, 3, 3, 3, 3, 3, 3, 3, 3, 3, 3]}
\vertices{20}
\edges{[[1, 6], [1, 7], [1, 20], [2, 8], [2, 9], [2, 10], [3, 11], [3, 12], [3, 13], [4, 14], [4, 15], [4, 16], [5, 17], [5, 18], [5, 19], [6, 9], [6, 18], [7, 10], [7, 19], [8, 11], [8, 20], [9, 12], [10, 13], [11, 14], [12, 15], [13, 16], [14, 17], [15, 18], [16, 19], [17, 20]]}
\pictures{G000055.png}
\comments 
  \begin{enumerate} 
 
    \item Let G be theta-connected, and not contain Petersen. If G contains Starfish then G is isomorphisc to Starfish., 
  \end{enumerate} 
 
\refs 
  \begin{enumerate} 
    \item N. Robertson, P. Seymour, R. Thomas, Excluded Minors in Cubic Graphs, 1995., 
  \end{enumerate} 
 
\owner{Aaron Bode.}

\graphname{G000056}
\graphtitle{Jaws}
\degrees{[3, 3, 3, 3, 3, 3, 3, 3, 3, 3, 3, 3, 3, 3, 3, 3, 3, 3, 3, 3]}
\vertices{20}
\edges{[[1, 2], [1, 6], [1, 13], [2, 3], [2, 9], [3, 4], [3, 8], [4, 5], [4, 10], [5, 6], [5, 11], [6, 7], [7, 8], [7, 12], [8, 20], [9, 10], [9, 14], [10, 16], [11, 12], [11, 17], [12, 19], [13, 14], [13, 18], [14, 15], [15, 16], [15, 20], [16, 17], [17, 18], [18, 19], [19, 20]]}
\pictures{G000056.png}
\comments 
  \begin{enumerate} 
 
    \item Let G be theta-connected, and not contain Petersen. If G contains Jaws then G is doublecross., 
  \end{enumerate} 
 
\refs 
  \begin{enumerate} 
    \item N. Robertson, P. Seymour, R. Thomas, Excluded Minors in Cubic Graphs, 1995., 
  \end{enumerate} 
 
\owner{Aaron Bode.}

\graphname{G000057}
\graphtitle{Triplex}
\degrees{[3, 3, 3, 3, 3, 3, 3, 3, 3, 3, 3, 3]}
\vertices{12}
\edges{[[1, 2], [1, 9], [1, 10], [2, 3], [2, 11], [3, 4], [3, 12], [4, 5], [4, 10], [5, 6], [5, 11], [6, 7], [6, 12], [7, 8], [7, 10], [8, 9], [8, 11], [9, 12]]}
\pictures{G000057.png}
\comments 
  \begin{enumerate} 
 
    \item Petersen, Triplex and Box are the only graphs minimal with the property of being dodecahedrally-connected and having crossing number > 1., 
  \end{enumerate} 
 
\links 
  \begin{enumerate} 
    \item \url{https://arxiv.org/pdf/1403.2118.pdf}, 
  \end{enumerate} 
 
\refs 
  \begin{enumerate} 
    \item N. Robertson, P. Seymour, R. Thomas, Excluded Minors in Cubic Graphs, 1995., 
  \end{enumerate} 
 
\owner{Aaron Bode.}

\graphname{G000058}
\graphtitle{Box}
\degrees{[3, 3, 3, 3, 3, 3, 3, 3, 3, 3, 3, 3, 3, 3]}
\vertices{14}
\edges{[[1, 2], [1, 4], [1, 10], [2, 3], [2, 13], [3, 6], [3, 7], [4, 5], [4, 12], [5, 6], [5, 13], [6, 9], [7, 8], [7, 10], [8, 9], [8, 14], [9, 12], [10, 11], [11, 12], [11, 14], [13, 14]]}
\pictures{G000058.png}
\comments 
  \begin{enumerate} 
 
    \item Petersen, Triplex and Box are the only graphs minimal with the property of being dodecahedrally-connected and having crossing number > 1., 
  \end{enumerate} 
 
\links 
  \begin{enumerate} 
    \item \url{https://arxiv.org/pdf/1403.2118.pdf}, 
  \end{enumerate} 
 
\refs 
  \begin{enumerate} 
    \item N. Robertson, P. Seymour, R. Thomas, Excluded Minors in Cubic Graphs, 1995., 
  \end{enumerate} 
 
\owner{Aaron Bode.}

\graphname{G000059}
\graphtitle{Antibox}
\degrees{[3, 3, 3, 3, 3, 3, 3, 3, 3, 3, 3, 3, 3, 3]}
\vertices{14}
\edges{[[1, 2], [1, 4], [1, 5], [2, 3], [2, 6], [3, 4], [3, 7], [4, 8], [5, 9], [5, 12], [6, 10], [6, 13], [7, 10], [7, 14], [8, 11], [8, 12], [9, 11], [9, 13], [10, 12], [11, 14], [13, 14]]}
\pictures{G000059.png}
\pictures{extra/G000059.png}
\links 
  \begin{enumerate} 
    \item \url{https://arxiv.org/pdf/1403.2118.pdf}, 
  \end{enumerate} 
 
\refs 
  \begin{enumerate} 
    \item N. Robertson, P. Seymour, R. Thomas, Excluded Minors in Cubic Graphs, 1995. Page 36., 
  \end{enumerate} 
 
\owner{Aaron Bode.}

\graphname{G000060}
\graphtitle{Window}
\degrees{[3, 3, 3, 3, 3, 3, 3, 3, 3, 3, 3, 3]}
\vertices{12}
\edges{[[1, 2], [1, 4], [1, 5], [2, 3], [2, 6], [3, 4], [3, 7], [4, 8], [5, 9], [5, 12], [6, 9], [6, 10], [7, 10], [7, 11], [8, 11], [8, 12], [9, 11], [10, 12]]}
\pictures{G000060.png}
\pictures{extra/G000060.png}
\links 
  \begin{enumerate} 
    \item \url{https://arxiv.org/pdf/1403.2118.pdf}, 
  \end{enumerate} 
 
\refs 
  \begin{enumerate} 
    \item N. Robertson, P. Seymour, R. Thomas, Excluded Minors in Cubic Graphs, 1995. Page 36., 
  \end{enumerate} 
 
\owner{Aaron Bode.}

\graphname{G000061}
\graphtitle{Drape}
\degrees{[3, 3, 3, 3, 3, 3, 3, 3, 3, 3, 3, 3, 3, 3]}
\vertices{14}
\edges{[[1, 2], [1, 4], [1, 5], [2, 3], [2, 6], [3, 4], [3, 7], [4, 8], [5, 12], [5, 13], [6, 9], [6, 10], [7, 10], [7, 11], [8, 11], [8, 14], [9, 11], [9, 13], [10, 12], [12, 14], [13, 14]]}
\pictures{G000061.png}
\pictures{extra/G000061.png}
\links 
  \begin{enumerate} 
    \item \url{https://arxiv.org/pdf/1403.2118.pdf}, 
  \end{enumerate} 
 
\refs 
  \begin{enumerate} 
    \item N. Robertson, P. Seymour, R. Thomas, Excluded Minors in Cubic Graphs, 1995. Page 36., 
  \end{enumerate} 
 
\owner{Aaron Bode.}

\graphname{G000062}
\graphtitle{Superbox}
\degrees{[3, 3, 3, 3, 3, 3, 3, 3, 3, 3, 3, 3, 3, 3, 3, 3]}
\vertices{16}
\edges{[[1, 2], [1, 5], [1, 11], [2, 3], [2, 12], [3, 4], [3, 8], [4, 5], [4, 13], [5, 14], [6, 7], [6, 10], [6, 11], [7, 8], [7, 12], [8, 9], [9, 10], [9, 13], [10, 14], [11, 15], [12, 15], [13, 16], [14, 16], [15, 16]]}
\pictures{G000062.png}
\pictures{extra/G000062.png}
\comments 
  \begin{enumerate} 
 
    \item Related to Box, G000058, 
  \end{enumerate} 
 
\links 
  \begin{enumerate} 
    \item \url{https://arxiv.org/pdf/1403.2118.pdf}, 
  \end{enumerate} 
 
\refs 
  \begin{enumerate} 
    \item N. Robertson, P. Seymour, R. Thomas, Excluded Minors in Cubic Graphs, 1995. Page 39., 
  \end{enumerate} 
 
\owner{Aaron Bode.}

\graphname{G000063}
\graphtitle{Drum}
\degrees{[3, 3, 3, 3, 3, 3, 3, 3, 3, 3, 3, 3, 3, 3]}
\vertices{14}
\edges{[[1, 2], [1, 3], [1, 4], [2, 5], [2, 6], [3, 7], [3, 11], [4, 8], [4, 12], [5, 9], [5, 11], [6, 10], [6, 12], [7, 8], [7, 14], [8, 13], [9, 10], [9, 14], [10, 13], [11, 13], [12, 14]]}
\pictures{G000063.png}
\pictures{extra/G000063.png}
\links 
  \begin{enumerate} 
    \item \url{https://arxiv.org/pdf/1403.2118.pdf}, 
  \end{enumerate} 
 
\refs 
  \begin{enumerate} 
    \item N. Robertson, P. Seymour, R. Thomas, Excluded Minors in Cubic Graphs, 1995. Page 41., 
  \end{enumerate} 
 
\owner{Aaron Bode.}

\graphname{G000064}
\graphtitle{Twinplex}
\degrees{[3, 3, 3, 3, 3, 3, 3, 3, 3, 3, 3, 3]}
\vertices{12}
\edges{[[1, 2], [1, 8], [1, 9], [2, 3], [2, 11], [3, 4], [3, 10], [4, 5], [4, 12], [5, 6], [5, 9], [6, 7], [6, 11], [7, 8], [7, 10], [8, 12], [9, 10], [11, 12]]}
\pictures{G000064.png}
\links 
  \begin{enumerate} 
    \item \url{https://arxiv.org/pdf/1403.2118.pdf}, 
  \end{enumerate} 
 
\refs 
  \begin{enumerate} 
    \item N. Robertson, P. Seymour, R. Thomas, Excluded Minors in Cubic Graphs, 1995., 
  \end{enumerate} 
 
\owner{Aaron Bode.}%%%%% Completed Graphlopedia Source File. %%% 